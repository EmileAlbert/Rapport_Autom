\section{Séance 4 : Internal Model Control}
\subsection{Introduction}
L'objectif de ce laboratoire est de découvrir une structure de réglage appelée structure de réglage par modèle interne ou "IMC" pour \textbf{I}nternal \textbf{M}odel \textbf{C}ontroller.\\

Le principe de cette structure de réglage est de considérer une transmisttance inverse à celle système en amont de celui-ci. De cette façon, la transmittance totale du système est égale 1. Ainsi, un échelon en entrée se retrouvera de manière identique en sortie.\\

\subsection{Internal Model Controller}
\subsubsection{Schéma simplifié}
\begin{figure}[H]
\centering
    
\tikzstyle{block} = [draw, fill=blue!20, rectangle, minimum height=3em, minimum width=6em]
\tikzstyle{sum} = [draw, fill=blue!20, circle, node distance=1cm]
\tikzstyle{input} = [coordinate]
\tikzstyle{output} = [coordinate]
\tikzstyle{tmp} = [coordinate]
\tikzstyle{pinstyle} = [pin edge={to-,thin,black}]

\begin{tikzpicture}[auto, node distance=2cm,>=latex']
% Blocks
       
\node [input, name=input] {};
\node [block, right of=input, node distance=4cm] (system-1) {$H^{-1}(s)$};
\node [block, right of=system-1, node distance=4cm] (system1) {$H(s)$};
\node [output, right of=system1, node distance=4cm] (output) {};
		
% Basic horizontal Flow
\draw [->] (input) -- node {$y_{sp}(s)$}(system-1); 
\draw [->] (system-1) -- node {$U(s)$} (system1);
\draw [->] (system1) -- node {$y_{pv}$}(output);
     


\end{tikzpicture}
\caption{Schéma simple de la structure de réglage \textbf{IMC}}
\end{figure}

Nous retrouvons sur le schéma selon le modèle de \textit{Vander Grinten}
\begin{itemize}
\item La fonction de transfert modélisant le système 
\begin{equation}
H(s) = \frac{k \cdot e^{-sT_{m}}}{(sT_{1} + 1)(sT_{2} + 1)} 
\end{equation}

\item La fonction de transfert inverse
\begin{equation}
H^{-1}(s) = (sT_{1} + 1) \cdot (sT_{2} + 1) \cdot \frac{1}{k}\cdot e^{sT_{m}}
\end{equation} 
\end{itemize}

Cette structure implique plusieurs choses :
\begin{itemize}
\item Inversion du temps mort\\
Sauf qu'on ne peut pas inverser le temps mort (système non causal)

\item Utilisation d'un modèle\\
Nous utilisons un modèle pour représenter le système. Cela suppose qu'il ne correspond pas exactement au comportement réel du système. Si le système est piloté avec les paramètres du modèle, il en résultera des erreurs. Ces erreurs s'additionneront avec les autres perturbations appliquées au système. 
 
\item Degré de la fonction de transfert\\
De plus, on a un degré plus élevé au numérateur (ce qui ipliquerait des impulsion de Dirac infinie en sortie) nous allons donc devoir mettre un dénominateur supérieur en filtrage pour équilibrer la fonction.
\end{itemize}

\subsubsection{Schéma usuel}
\begin{figure}[H]
\centering
    
\tikzstyle{block} = [draw, fill=blue!20, rectangle, minimum height=3em, minimum width=6em]
\tikzstyle{sum} = [draw, fill=blue!20, circle, node distance=1cm]
\tikzstyle{input} = [coordinate]
\tikzstyle{output} = [coordinate]
\tikzstyle{tmp} = [coordinate]
\tikzstyle{pinstyle} = [pin edge={to-,thin,black}]

\begin{tikzpicture}[auto, node distance=2cm,>=latex']
% Blocks
       
\node [input, name=input] {};
\node [sum, right of=input] (sum) {};
\node [block, right of=sum, node distance=3cm] (system-1) {$H^{-1}(s)$};
\node [sum, right of=system-1, node distance=3cm] (point) {};
\node [sum, right of=point, node distance=0.5cm] (sum2) {};
\node [block, right of=sum2, node distance=3cm] (system1) {$H(s)$};
\node [sum, right of=system1, node distance=3cm] (sum4) {}; 
\node [sum, right of=sum4, node distance=0.5cm] (point1) {};
\node [output, right of=sum4, node distance=1.5cm] (output) {};

\node [tmp, below of=point, node distance=3cm] (link_tmp){};       
\node [block, below of=system1, node distance=3cm] (systemm) {$H_{m}(s)$};    
\node [sum, below of=sum4, node distance=3cm] (sum5) {}; 
		
\node [tmp, below of=sum5, node distance=1cm] (link_tmp1){};
\node [tmp, below of=sum, node distance=4cm] (link_tmp2){};
		
\node [tmp, above of=sum2, node distance=2cm] (d'){};
\node [tmp, above of=sum4, node distance=2cm] (d){};
		
% Basic horizontal Flow
\draw [->] (input) -- node {$y_{sp}$}(sum);
\draw [->] (sum) -- (system-1);   
\draw      (system-1) -- node {$u$} (point);
\draw [->] (point) -- (sum2);
\draw [->] (sum2) -- (system1);
\draw      (system1) -- (sum4);
\draw [->] (sum4) -- (point1);
\draw [->] (point1) -- node {$y_{pv}$}(output);
     
\draw [->] (link_tmp) -- (systemm);
\draw [->] (systemm) -- (sum5);    
        
\draw      (link_tmp1) -- (link_tmp2);
        
% Basic vertical Flow
\draw [->] (d') -- node {$d'$}(sum2);
\draw [->] (d) -- node {$d$}(sum4);
		
\draw      (point) -- (link_tmp);
\draw [->] (sum4) -- (sum5);
        
\draw      (sum5) -- node {$\widehat{d}$}(link_tmp1);
\draw [->] (link_tmp2) -- (sum);
  
\end{tikzpicture}
\caption{Schéma de la structure de réglage \textbf{IMC}}
\end{figure}

Dans cette configuration, pour régler le problème du degré du numérateur plus élevé dans la fonction de transfert inverse, on ajoute un terme $T_{f}$. On notera que l'expression du temps mort à aussi été supprimé en rapport au problème présenté dans la section précédente.\\

La fonction de transfert inverse devient alors 
\begin{equation}
H_{mf}^{-1}(s) = \frac{(sT_{1} + 1) \cdot (sT_{2} + 1)}{(sT_{f} + 1)^{2}} \cdot \frac{1}{k}
\end{equation}

\subsection{IMC avec Simulink}
En pratique nous utilisons simulink pour contrôler le système sous une vue graphique. Le diagramme présenté plus haut devient alors la figure suivante.
\begin{figure}[H]
\includegraphics[width = \textwidth]{../Pictures/IMC/simulink_lab4.png}
\end{figure}
\subsection{Graphe de la réponse du système}
Lors de l'essai représenté à la figure ci-dessous, nous observons très clairement 3 périodes.
\begin{figure}[H]
\includegraphics[width= \textwidth]{../Pictures/IMC/IMC_glob.png}
\end{figure}
Le première commence en zéro jusqu'à 580 sec et représente l'initialisation du système. Ceci correspond la réaction du système à partir du moment où l'on applique le régulateur et nous observons que la valeur s'éloigne très fortement de la valeur de consigne avant de remonter pour venir se plaquer contre cette dernière. Cette chute rapide est due à \#\# je sais plus .
\begin{figure}[H]
\includegraphics[width= 0.8\textwidth]{../Pictures/IMC/amorce_sys.png}
\end{figure}
La deuxième partie se déroule de 550 sec à 780 sec. Nous appliquons durant cette période un échelon de 10\% afin de vérifier la performance du régulateur en suivit de consigne. Remarquons que les performances du systèmes à ce sujet son excellente ! Nous remarquons en effet que le système retrouve un situations d'équilibre vers la 720ième seconde soit 140 secondes après le début de l'impulsion là où notre régulateur PID en mettait \#\#.
\begin{figure}[H]
\includegraphics[width= 0.8\textwidth]{../Pictures/IMC/suivit_consigne.png}
\end{figure}
Enfin, la troisième zone va de la 760ième seconde jusqu'à la fin. Cette zone correspond à l'insertion  d'une perturbation dans le système. Dans notre cas, l'action sur l'intérupteur de la résistance chauffante a provoqué une soudaine montée en température que notre régulateur a eu beaucoup de mal à neutraliser. La valeur de la résistance ayant en effet augmenté de 4\% avant d'être atténuée contre \#\# \% pour le régulateur PID, nous pouvons conclure que la réjection de perturbations est la grosse faiblesse de ce genre de régulateur. 
\begin{figure}[H]
\includegraphics[width= 0.8\textwidth]{../Pictures/IMC/rej_perturb.png}
\end{figure}
Ceci est du au fait que \#\#.
\subsection{Amélioration de la réjection de perturbation}

\#\# La manip a merdé et je me souvient plus vrmt de pk ni de ce qu'on était sensé obtenir. à compléter !!s