\section{Séance 1 : Identification}
\subsection{Introduction}
\begin{center}
\textit{Une bonne régulation ne peut se faire sans un bon modèle}
\end{center}

Les régulateurs sont destinés à opérer sur des systèmes physiques. 
Les réponses qu'ils doivent fournir pour une bonne régulation s'appuient sur 
leur synthèse et les paramètres qui en découlent. 
Ces paramètres doivent correspondre au plus près à la réalité physique du système. 
Cette réalité est exprimée au travers du modèle mathématique associé à ce dernier. 
Ce modèle correspond à la fonction de transfert du système, 
autrement dit à la relation liant son (ses) entrée(s) à sa (ses) sortie(s).\\

\subsection{Système}
Dans le cadre de ce laboratoire, le système que nous allons étudier est un 
canon à chaleur didactique 'LTR700'. Il est constitué d'une résistance chauffante, 
d'un ventilateur et d'une sonde de température à la sortie.

De plus, le système possède deux mécanismes permettant d'introduire des perturbations:
\begin{enumerate}
    \item Un clapet pouvant modifier l'entrée d'air du canon
    \item Un switch permettant de court-circuiter une partie de la résistance
\end{enumerate}


\fig{S1_Systeme.png}{1}{Schéma du système}


\subsubsection{Entrées - Sorties}
L'entrée $u$ du système correspond au courant $i$ injecté dans la résistance, 
directement lié à la puissance thermique dissipée à l'intérieur du canon ($P = R \cdot I^{2}$).\\
 
La sortie $y$ du système, quant à elle, correspond à la température de l'air $T^\circ$
mesuré à la sortie du canon à l'aide d'une sonde de température 'PT100'.\\ 

\begin{figure}[H]	
    \centering
    \tikzstyle{int}=[draw, fill=blue!20, minimum size=2em]
    \begin{tikzpicture}[auto, node distance=2cm,>=latex']
        \node[int] (system) {$H(s)$};
        \node (input) [left of=system,node distance=2cm, coordinate] {system};
        \node [coordinate] (output) [right of=system, node distance=2cm]{};
        \path[->] (input) edge node {$i$} (system);
        \path[->] (system) edge node {$T^\circ$} (output);
    \end{tikzpicture}
    \caption{Schéma bloc}
\end{figure}


\subsubsection{Linéarisation du système}
Le système n'est pas linéaire puisque, premièrement, l'équation de la puissance en fonction
du courant ($P = R \cdot I^2$) introduit déjà un exposant d'ordre 2. 
Deuxièmement, l'échange de chaleur entre la résistance et l'air se fait par convection, 
mais également par rayonnement ($P = \epsilon S \sigma T^4$), ce qui introduit des termes
de puissance 4.\\

Il faut donc linéariser le système. On travaillera en petite variations afin que
l'hypothèse de linéarité soit valide.
 


\subsection{Modélisation vs. identification}
Avant de pouvoir réguler le système, il nous faut trouver un modèle dynamique fiable
du système, qui nous permettera de savoir ou estimer comment le système réagira aux
variations de ses entrées.\\

Il existe deux méthodes principales pour trouver le modèle dynamique d'un système: 
\subsubsection{La modélisation}
La modélisation, consiste à utiliser les équations différentielles régissant le système.
On peut ainsi trouver la fonction de transfert en exprimant la relation des sorties aux 
entrées dans le domaine de Laplace. La modélisation d'un système nécessite cependant de
connaitre toutes les équations différentielles, ainsi que leurs coefficients.
Ceci n'est pas toujours facile, ni même possible. Dans ce cas, nous pourrons avoir recourt
a des méthodes d'identification du système par voie expérimentale.

\subsubsection{L'identification}
L'identification d'un système consiste à le considérer comme une boite noire et à réaliser
des mesures pertinentes sur celui-ci afin de pouvoir en déduire sa fonction de transfert.
Une mesure qui est souvent réalisée afin d'identifier le système est la réponse indicielle.
Elle consiste à mettre un échelon en entrée et mesurer la sortie. A partir de la réponse,
il est possible d'approximer la fonction de transfert du système.\\

C'est ce que nous alons faire durant cette séance. Les objectifs de cette 
première séance consiste à comparer trois méthodes de modélisations 
différentes pour le système physique, de trouver un modèle optimal en exécutant une routine
d'optimisation dans Matlab et finalement de synthétiser par méthode directe un 
régulateur PID (méthode de Chien-Hrones-Reswick).


\subsection{Essai indiciel}
\subsubsection{Description et conditions de l'essai}
Dans le but de déterminer la fonction de transfert du système, nous allons donc appliquer un échelon à celui-ci.\\

Le système étant d'un degré supérieur à un, nous allons travailler autour d'un point de fonctionnement pour pouvoir considérer la réponse du système linéaire sur cet intervalle. Ce point de fonctionnement est 50\% de la température et donc un intervalle égal à $[45\% - 55\%]$

L'échelon sera appliqué sur la totalité de la plage. Le système étant considéré comme linéaire, la raison de ce choix n'est pas d'augmenter la précision d'une approximation, mais plutôt de minimiser l'erreur induite par le bruit. En effet, un échelon plus grand permet de réduire le ratio signal/bruit.\\

\paragraph{Condition de l'essai}
\begin{itemize}
\item Le système didactique LTR700 utilisé est le numéro 3
\item Position du clapet - Ouvert
\item Position de l'interrupteur de perturbation - 0
\end{itemize}

\paragraph{Identification}
\begin{itemize}
\item Mode régulateur - MANU
\item Échelon de température de l'air- de 45\% à 55\% - $\Delta = 10\%$
\item Mesure de la température de l'air en sortie de $39.5\%$ à $49.5\%$ - $\Delta = 10\%$

\fig{S1_RepInd}{0.55}{Réponse indicielle du système}

\begin{itemize}
\item On trouve la valeur de a graphiquement grâce à la formule $a = \frac{a'}{\Delta y} = \frac{2.52}{9.78} = 0.2577$

\item $T_{U}$ est trouvé graphiquement et vaut 20.83 

\item $T_{G}$ est trouvé graphiquement et vaut 145.83

\item $T_{1}$ est trouvé graphiquement et vaut 58.33

\item $T_{2}$ est trouvé graphiquement et vaut 75
\end{itemize}

Les graphiques manuelles utilisés pour déterminer ces constantes se trouvent en annexe. 

\item Gain statique $K = \frac{\Delta y}{\Delta u} = \frac{13.0}{13.04} = 0.970 $
\end{itemize}

\subsubsection{Modèle de Broïda}
On modélise la fonction de transfert d'un système d'ordre n à l'aide d'un modèle du premier ordre combinée à un retard pur
\begin{equation}
H_{B}(s) = \frac{K}{1 + Ts}*e^{-T_{m}s}
\end{equation}

où 
\begin{itemize}
\item K est le gain statique [sans unité]
\item $T$ est la constante de temps [s]
\item $T_{m}$ est le temps mort [s]
\end{itemize}

Pour notre système nous obtenons suivant les formules et les valeurs trouvées graphiquement 
\begin{itemize}
\item $T = 5.5 (t_{2} - t_{1}) = 5.5 (75 - 58.33) = 91.685$
\item $T_{m} = 2.8 t_{1} - 1.8 t_{2} = 2.8*58.33 - 1.8*75 = 28.324$
\end{itemize}

Ce qui donne 
\begin{equation}
H_{B}(s) = \frac{0.970}{1 + 91.685s }*e^{-28.324s}
\end{equation}

\subsubsection{Modèle de Vander Grinten}
Modélisation du système grâce à un modèle du second ordre avec deux pôles distincts et un temps mort

\begin{equation}
H_{V}(s) = \frac{K e^{-s T_{m}}}{(sT_{1} + 1) (sT_{2} + 1)}
\end{equation}

où 
\begin{itemize}
\item K est le gain statique [sans unité]
\item $T_{1}$ est la première constante de temps [s]
\item $T_{2}$ est la deuxième constante de temps [s]
\item $T_{m}$ est le temps mort [s]
\end{itemize}

Pour notre système nous obtenons suivant les formules et les valeurs trouvées graphiquement 
\begin{itemize}
\item $T_{1} = T_{G} \frac{3ae - 1}{1 + ae} = 145.83 \frac{3*0.2577*e - 1}{1 + 0.2577*e} = 94.445$
\item $T_{2} = T_{G} \frac{1 - ae}{1 + ae} = 145.83 \frac{1 - 0.2577*e}{1 + 0.2577*e} = 25.693$
\item $T_{m} = T_{U} \frac{T_{1}T_{2}}{T_{1} + 3T_{2}} = 20.83 \frac{94.445*25.693}{94.445 + 3*25.693} = 6.683$
\end{itemize}

\subsubsection{\textcolor{red}{Modèle de Strejc}}
Ce modèle identifie un ordre n quelconque à condition d'avoir tous les pôles identiques
\begin{equation}
H_{S}(s) = \frac{K e^{-s T_{m}}}{(sT + 1)^{n}}
\end{equation}

où 
\begin{itemize}
\item K est le gain statique [sans unité]
\item $T$ est la constante de temps [s]
\item $T_{m}$ est le temps mort [s] 
\item n est l'ordre du modèle
\end{itemize}

Pour notre système, les valeurs calculées pour définir les paramètres du modèle de Strejc sont
\begin{itemize}
\item $\frac{T_{U}}{T_{G}} = \frac{20.83}{145.83} = 0.143$
\item $\frac{T_{G}}{T} = \frac{145.83}{91.685} = 1.591 $
\textcolor{red}{Avec $T = 5.5 (t_{2} - t_{1}) = 5.5 (75 - 58.33) = 91.685$ ?}
\end{itemize}

L'ordre est alors déterminé en fonction du tableau fournis dans les notes de laboratoires 
\fig{S1_Stejc_table.png}{1}{}

\textbf{Remarque :}Dans le cas où les valeurs calculée tombent entre les valeurs énoncées dans le tableau, on prendra la valeur inférieure pour définir l'ordre. En effet, la courbe correspondante à l'ordre inférieure pourra être rectifiée avec un temps mort pur pour se rapprocher de la courbe réelle, au contraire de la courbe correspondante à l'ordre supérieur. 

\subsubsection{Optimisation des modèles}
L'objectif de cette méthode est de trouver un modèle optimal par recherche algorithmique par fonction de coût. Les modèles trouvés précédemment sont donc améliorés par la routine \url{fminsearch} de MATLAB.


\subsection{Observations}
Les 3 courbes des modèles utilisés pour modéliser le systèmes sont reprises ci-dessous

\fig{S1_Superposition_model}{0.55}{}

A l'aide de la fonction matlab d'optimisation, nous avons déterminé un modèle idéal et ses paramètres associés pour le système étudié

\fig{S1_Optimisation}{0.7}{}

De cette optimisation découle 4 paramètres :
\begin{itemize}
\item $K_{P}$ = 0.98
\item $T_{1}$ = 114.74
\item $T_{2}$ = 7.12
\item $T_{m}$ = 14.97
\end{itemize}

\subsection{Synthèse de régulateur par la méthode de Chien - Hrones - Reswick}
La méthode de Chien - Hrones - Reswick est un méthode de synthèse directe de régulateur PID. Le type de régulateur (P, PI ou PID) est choisi en fonction de la réglabilité du système étudié et ses paramètres sont ensuite calculée en fonction des constantes de temps observées. 

\begin{itemize}
\item Réglabilité $\rho = \frac{T_{G}}{T_{U}} = \frac{145.83}{20.83} = 7.001$\\
Selon les critères de la méthodes, le régulateur adapté est un régulateur PI ($10 > \rho >= 7$)

On trouve alors 
\begin{itemize}
\item $K_{BO} = \frac{\rho}{2.9} = \frac{7.001}{2.9} = 2.414$
\item $K_{r} = \frac{K_{BO}}{K_{s}} = \frac{2.414}{0.97} = 2.489$
\item $T_{i} = 1.2T_{G} = 1.2*145.83 = 174.996$
\end{itemize}
\end{itemize}

Si on impose de travailler avec un régulateur PID on trouve alors 
\begin{itemize}
\item $K_{BO} = \frac{\rho}{1.7} = \frac{7.001}{1.7} = 4.118$
\item $K_{r} = \frac{K_{BO}}{K_{s}} = \frac{4.118}{0.97} = 2.245$
\item $T_{i} = T_{G} = 145.83$

\textcolor{red}{\item $T_{d} = (0.5...1)T_{U} = (0.5...1)*20.83 = $
\item $T_{f} = \frac{T_{d}}{10...20} = $}
\end{itemize}
