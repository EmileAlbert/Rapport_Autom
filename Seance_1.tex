\section{Séance 1 : Identification}
\subsection{Introduction}
\begin{center}
\textit{Une bonne régulation ne peut se faire sans un bon modèle}
\end{center}

Les régulateurs sont destinés à opérer dans des systèmes physiques. Les réponses qu'ils doivent fournir pour une bonne régulation s'appuient sur leur synthèse et les paramètres qui en découlent. Ces paramètres doivent correspondre au plus près à la réalité physique du système. Cette réalité est exprimée au travers du modèle mathématique associé à ce dernier. Ce modèle correspond à la fonction de transfert du système, autrement dit à la relation liant son (ses) entrée(s) à sa (ses) sortie(s).\\

\subsubsection{Système}
Dans le cadre de ce laboratoire, le système est un canon à chaleur. Il est constitué d'une résistance chauffante, d'en ventilateur ainsi que d'un clapet pouvant modifier l'entrée d'air du canon.

\fig{Systeme.png}{1}{Schéma du système}

L'entrée du système correspond au courant injecté dans la résistance, directement relié à la puissance thermique dissipée dans le canon ($P = R*I^{2}$).\\
 
La sortie du système, quant à elle, correspond à la température de l'air à la sortie du canon mesurée au moyen d'un capteur industriel \textcolor{red}{LTR700}.\\ 

Au vu de ces considérations, le système ne peut pas être représenté par un système du premier ordre. En effet, l'équation de puissance dissipée et de transfert de chaleur par rayonnement apporte des termes de puissance deux et quatre.\\

Les objectifs de cette première séance sont d'une part comparer trois modélisations différentes pour le système physique et de trouver un modèle optimal et d'autre part de synthétiser par méthode directe un régulateur PID (méthode de Chien-Hrones-Reswick). 

\textcolor{red}{Schéma bloc des E/S et de la fonction de transfert}

\subsection{Essai indiciel}
\subsubsection{Description et conditions de l'essai}
Dans le but de déterminer la fonction de transfert du système, nous allons donc appliquer un échelon à celui-ci.\\

Le système étant d'un degré supérieur à un, nous allons travailler autour d'un point de fonctionnement pour pouvoir considérer la réponse du système linéaire sur cet intervalle. Ce point de fonctionnement est 50\% de la température et donc un intervalle égal à $[45\% - 55\%]$

L'échelon sera appliqué sur la totalité de la plage. Le système étant considéré comme linéaire, la raison de ce choix n'est pas d'augmenter la précision d'une approximation, mais plutôt de minimiser l'erreur induite par le bruit. En effet, un échelon plus grand permet de réduire le ratio signal/bruit.\\

\textbf{Condition de l'essai}
\begin{itemize}
\item \textcolor{red}{LTR700}
\item Position du clapet - Ouvert
\item Position de l'interrupteur de perturbation - 0
\end{itemize}

\textbf{Identification}
\begin{itemize}
\item Mode régulateur - MANU
\item Échelon de température de l'air- de 45\% à 55\% - $\Delta = 10\%$
\item \textcolor{red}{Mesure de la température de l'air en sortie - de ... à ... - $\Delta =$}  

\item \textcolor{red}{Courbe MATLAB}

\item \textcolor{red}{Gain statique $k = \frac{\Delta y}{\Delta u} = $}
\end{itemize}

\subsubsection{Modèle de Broïda}
\begin{equation}
H(s) = \frac{K}{1 + T_{S}}*e^{-T_{m}s}
\end{equation}

\textcolor{red}{Avec}
\begin{itemize}
\item K 
\item $T_{S}$
\item $T_{m}$
\item s
\end{itemize}



\subsubsection{Modèle de Vander Grinten}
\begin{equation}
H(s) = \frac{K e^{-s T_{m}}}{(sT_{1} + 1) (sT_{2} + 1)}
\end{equation}

\textcolor{red}{Avec}
\begin{itemize}
\item K 
\item $T_{S}$
\item $T_{1}$
\item $T_{2}$
\item s
\end{itemize}

\subsubsection{Modèle de Stejc}
\begin{equation}
H(s) = \frac{K e^{-s T_{m}}}{(sT + 1)^{n}}
\end{equation}

\textcolor{red}{Avec}
\begin{itemize}
\item K 
\item $T_{S}$
\item $T_{1}$
\item $T_{2}$
\item s
\end{itemize}

\fig{Stejc_table.png}{1}{\textcolor{red}{Légende}}


\subsubsection{Optimisation des modèles}
L'objectif de cette méthode est de trouver un modèle optimal par recherche algorithmique par fonction de coût. Les modèles trouvés précédemment sont donc améliorés par la routine \url{fminsearch} de MATLAB.


\subsection{Commentaires}
conclusion temporaire - superposition des courbes 

\subsection{Synthèse de régulateur par la méthode de Chien - Hrones - Reswick}

