\section{Séance 2 : Etude du régulateur PID}
\subsection{Introduction}

\subsection{Essais}
Attention, le gain utilisé ici comme gain K0 n'est pas le gain trouvé au laboratoire 1 par la méthode CHR. Ce dernier étant trop faible (effet de l'imprécision sur la détermination de $T_{U}$ et $T_{G}$) on a décidé de prendre $K0 = 1.5 \cdot$ Gain du premier laboratoire. Cette valeur est celle utilisée pour l'ensemble des essais de ce laboratoire. 

Attention Ko = 1.5 Gain trouvé au labo 1 - explication
\begin{itemize}
\item Essais 0 - s2e0 - 0.66Ko Tio Tdo\\
Comme expliqué dans l'introduction, le K0 utilisé ici un multiple de celui calculé au premier laboratoire. Cette essais ci visant à observer le comportement des valeurs du premier laboratoire, nous avons ajouter le coefficient de 0.66 pour retrouver le K0 du premier laboratoire.

\item Essais 1 - s2e1 - K0 Ti0 Td0

\item Essais 2 - pas fait - 0.5K0 Ti0 Td0 

\item Essais 3 - s2e3 - K0- 100000 Td0

\item Essais 4 - s2e4 - K0 - 100000 0.001

\item Essais 5 - s2e5 - 0.5K0 - 100000 0.001

\item Essais 6 - s2e6 - K0 - Ti0 2Td0

\item Essais 7 - s2e7 - K0 - Ti0 4Td0

\item Essais 8 - s2e8 - 8K0 - 100000 0.001





\end{itemize}