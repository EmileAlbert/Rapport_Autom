\section{Séance 2 : \'Etude du régulateur PID}
\subsection{Introduction}
La première séance avait pour but d'identifier le système thermique que nous
devons réguler. L'identification du système a permit un premier ensemble de gains
pour un régulateur de type PID en utilisant la méthode de 
\textbf{Chien - Hrones - Reswick}. Les gains trouvés ne sont en aucun cas un optimum,
mais un bon point de départ pour commencer à les optimiser.\\

Cette séance continue à batir sur ce que nous avons trouvé durant la première séance,
en observant l'influence des gains d'un régulateur PID sur notre système. 
Nous alons donc faire plusieurs essais en faisant varier les gains et en comparant
les réponses du sytème régulé, afin de pouvoir visualiser et tirer des conclusions
sur les effets des différents gains du régulateur PID.
 

\subsection{Essais}
A la suite de la première séance, nous avons trouvés les valeurs de gains suivants
à l'aide de la méthode de Chien - Hrones - Reswick:

\begin{center}
    $K_0  = 4.118$
    \hspace{1cm}
    $T_{i0} = 145.830$
    \hspace{1cm}
    $T_{d0} = 15.662$
    \hspace{1cm}
    $T_{f0} = 1.041$
\end{center}

\paragraph{\textcolor{red}{Attention:}}\mbox{}\\
Le \textcolor{red}{gain proportionnel} $K_0$ que nous avons trouvé à l'aide de la méthode de 
Chien - Hrones - Reswick était relativement faible, du à l'imprécision 
sur la détermination graphique de $T_{U}$ et $T_{G}$. Durant le premier essai,
décrit dans la section \ref{essai-0}, nous avons constaté que le système mettait
énormément de temps à se stabiliser. Nous avons mit plus de \textcolor{red}{35 min}
a terminer cet essai. Nous avons donc décider d'augmenter ce coefficient pour le
reste des manipulations, afin de pouvoir terminer les 9 essais dans les temps.\\

Soit si on indique le gain trouvé initialement par $K_0'$, nous avons décider de 
prendre comme gain $$K_0 = 1.5 \cdot K_0' = 6.177$$


\subsubsection{Essai 0 - s2e0 - 0.66Ko Tio Tdo} \label{essai-0}
Comme mentionné dans le paragraph précédent, le premier essai s'est fait avec
la valeur initiale $K_0'$ trouvée par la méthode de Chien - Hrones - Reswick.
Cependant comme nous avons corrigé le gain de base $K_0$ après cette essai, nous
utiliseront cette essai, avec $K = 0.66 \cdot K_0$, comme l'essai qu'il fallait
avec un gain $K = 0.5 \cdot K_0$.

\subsubsection{Essai 1 - s2e1 - K0 Ti0 Td0}

\subsubsection{Essai 2 - pas fait - 0.5K0 Ti0 Td0 }

\subsubsection{Essai 3 - s2e3 - K0- 100000 Td0}

\subsubsection{Essai 4 - s2e4 - K0 - 100000 0.001}

\subsubsection{Essai 5 - s2e5 - 0.5K0 - 100000 0.001}

\subsubsection{Essai 6 - s2e6 - K0 - Ti0 2Td0}

\subsubsection{Essai 7 - s2e7 - K0 - Ti0 4Td0}

\subsubsection{Essai 8 - s2e8 - 8K0 - 100000 0.001}


\subsection{Comparaison des essais}
\textcolor{red}{Diagramme de Bode - Diagramme temporelle\\}

\textcolor{red}{Stabilité, temps de réponse, dépassement, forme de l’action (énergique, bruitée), action maximale demandée, marges de gain et de phase en fonction de K, Ti et Td.}

\subsection{Régulateur optimisé}
Sur base des observations précédentes, nous avons pu intuitivement modifier les valeurs des trois paramètres du régulateur PID pour tenter de l'optimiser.\\

\textcolor{red}{Présentation de ses résultats}
